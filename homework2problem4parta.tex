\subsection{Part A}

We are tasked with building a numerical integrator such that we can determine the linear growth factor at any redshift:

\begin{align}
D(z) &= {5\Omega_{m} H_{0}^{2} \over 2} H(z) \int_{z}^{\infty} {1+z' \over H^{3}(z')} \dd z'
\end{align}

The improper integral is addressed by using a variable substitution proffered in Mark Newman's Computational Physics textbook:

\begin{align}
z &= {x-a \over 1+x-a}, \\
\int_{a}^{\infty} f(x) \dd x &= \int_{0}^{1} {1\over (1-z)^{2}} \, f\left({z \over 1-z} + a\right) \dd z.
\end{align}

However, the integrand is undefined at $z = 1$ so we introducing a fudge factor $\epsilon = 10^{-14}$ to the integration limits. The desired accuracy is achieved by implementing Simpson's integration with $N$ points, comparing it to the same integration with $2N$ points, and continuing in that fashion until the difference is less than $10^{-5}$.

\lstinputlisting{homework2problem4parta_result.txt}

\lstinputlisting{homework2problem4parta.py}
