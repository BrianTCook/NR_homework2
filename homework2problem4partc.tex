\subsection*{Part C}

Each particle gets the following position and momentum update for the relevant scale factor $a$:

\begin{align}
\mat{x} &= \mat{q} + D(a) \mat{S}(\mat{q}), \\
\mat{p} &= -(a-\Delta a /2)^{2}\dot{D}(a-\Delta a/2) \mat{S}(\mat{q}).
\end{align}

To compute the vectors $\mat{S}(\mat{q})$ for each particle I populate two arrays with values of $c_{k}$ (as defined in the instructions) and then take the \textit{inverse} Fourier transform. I then define a particle class that stores four vectors: $\mat{x}, \mat{p}, \mat{q}, \mat{S}(\mat{q})$. I compute the cofactors for the position and momentum vectors at each scale factor and then update the position/momentum vectors for each particle appropriately. I was having issues with the given ffmpeg command so I used convert instead to generate a gif that shows 90 frames in 3 seconds, where each of the frames corresponds to an appropriate scale factor.

I anticipate that the level of collapse is not what was expected, and this can probably be attributed to either an erroneous computation of $D$ or initializing the values of $\mat{q}$ or $\mat{S}(\mat{q})$ incorrectly.

\lstinputlisting{homework2problem4partc.py}

\clearpage

\begin{figure}[h]
    \centering
    \includegraphics{homework2problem4partcfigure1.pdf}
    \caption{Plotting the $y$ position of the first 10 particles.}
    \label{fig:24c1}
\end{figure}

\begin{figure}[h]
    \centering
    \includegraphics{homework2problem4partcfigure2.pdf}
    \caption{Plotting the $y$-component of the momentum $p_{y}$ of the first 10 particles.}
    \label{fig:24c2}
\end{figure}

\clearpage
